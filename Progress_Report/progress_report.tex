\documentclass[12pt,a4paper,twoside,openright]{report}
\usepackage{preamble} % imported packages/macros/styling

\usepackage[none]{hyphenat}

\vspace{2mm}

\chapter*{Progress Report}

\begin{document}

{
\begin{tabular}{ll}
  Name:          & \bf Brace Godfrey                       \\[3pt]
  Email:         & \bf dg681@cam.ac.uk                  \\[3pt]
  Project Title: & \bf Continuous Pursuit-Evasion with Aerodynamic Constraints \\[3pt]
  Day-To-Day Supervisor:    & \bf Ramsey Faragher                     \\[3pt]
  Marking Supervisor:    & \bf Alastair Beresford                     \\[3pt]
  Director of Studies:           & \bf Ramsey Faragher                  \\[3pt]
\end{tabular}
}

\vspace{5mm}

\subsection*{Summary}

The aim of my project is to create a kinematically accurate simulation of a continuous pursuit-evasion scenario involving multiple aircraft. The simulation is implemented in Python with a Markov decision process (MDP) determining the optimal action for each aircraft at each timestep. I am currently on schedule with my proposal timeline and have met all major milestones.

The simulation framework is complete including discrete timesteps, configurable simulation properties and output visualisations. The FastMDP algorithm has been implemented to compute optimal actions based on reward functions, and the kinematic models have been tentatively implemented up to 3D aircraft with aerodynamic constraints.

I have made some minor changes to my original timeline, primarily by swapping the implementation order of the kinematic models and decision processes so that I developed the MDP framework first. This allowed me to begin with a simpler simulation model and progress incrementally from 1D to 2D to 3D, making it easier to identify and resolve issues with the decision-making process without the added complexity of full 3D aerodynamics. The reward functions for the MDP depend on the dimensionality of the simulation, so I dedicated additional time to developing and tuning these for each kinematic model. To complete the project core I plan to further refine the reward functions to drive desired pursuit-evasion behaviours and finalise the 3D aerodynamic model. I have allocated slack time at the end of February for validation of the simulation results and milestone catch-up if necessary, and then I will progress onto exploring extensions and drafting the final report.

I have also slightly reconsidered my final evaluation and success criteria. Through developing the simulation, I have found that simulation performance and implementation efficiency may be useful evaluation points. I will evaluate the project based on simulation performance and result quality. This will include assessing the impact and reasoning of system-level optimisations such as caching, memory use, and NumPy/Numba acceleration.

\end{document}